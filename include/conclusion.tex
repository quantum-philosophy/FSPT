In this paper, we accented the need for developing simulators of electronic
systems with data-driven applications in mind. We argued that the techniques
capitalizing on learning from data have special requirements, and that the
state-of-the-art simulators are unable to fulfill them due to prohibitively
large simulation times. Acknowledging the importance of power and temperature
the design of multiprocessor systems, we developed a methodology for a fast
generation of synthetic power and temperature traces that preserve the
idiosyncrasies of their real-life counterparts. Last but not least, we
implemented a toolchain that embodies the presented approach.

We hope that this work will enable new and assist ongoing studies by making it
easier to explore the potential of novel or revived but potentially
data-demanding techniques for analysis, prediction, and management of electronic
systems such as artificial neural networks with deep architecture.

Lastly, we would like to remind that the toolchain, repositories with reference
arrival and workload patterns, and other supplementary materials are publicly
available \cite{sources}.
