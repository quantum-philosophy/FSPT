Consider an in-design or in-production computer system. The system is composed
of a number of processing elements and is capable of performing a number of
operations. The system receives a stream of user requests or jobs, which it is
supposed to process. Each job implies a certain amount of work to be done and,
thus, a certain amount of power to be consumed and a certain amount of heat to
be dissipated.

Consider now a research project targeted at developing a resource manager for
the system under consideration. Since the importance of power and temperature is
well understood, the solution is required to take these two quantities into
account. Suppose further that the resource manager is to be based on a technique
that learns from the data available at runtime.

Our goal is to develop a toolchain for generating power and temperature profiles
of the system in order to provide the project with plenty of data to experiment
with. The generated profiles should preserve the particularities of job arrivals
and program executions that are present in real life, which is what makes the
subsequent learning meaningful. In addition, the generation process should be
substantially faster than traditional simulations, which is what makes the work
worth doing.
