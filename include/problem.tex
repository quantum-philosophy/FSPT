Consider an in-design or in-production electronic system. The system is composed
of a number of processing elements and is capable of performing a number of
operations. The system receives a stream of requests or jobs, which it is
supposed to process. Each job implies a certain amount of work to be done and,
therefore, a certain amount of power to be consumed and a certain amount of heat
to be dissipated.

Consider now a hypothetical research project targeted at developing a management
strategy for the system at hand, such as a scheduling policy. Since the
importance of power and temperature is well understood, which was motivated in
\sref{introduction}, the solution is required to take into account the two
quantities. Suppose further that the solution is to be based on a technique that
learns from the power and/or temperature data available on the chip, which was
motivated in \sref{motivation}.

Our goal is to develop a methodology for fast generation of realistic power and
temperature profiles of the system in order to provide the research project with
plenty of data to experiment with. Hence, the requirements to the data that we
would like to produce are as follows. First, the generated profiles should
preserve the particularities of job arrivals and program executions that are
present in real life, which is what makes the subsequent learning meaningful.
Second, the generation should be substantially faster than performing
traditional simulations, which is what makes our work worth doing.

In addition, we would like to construct a toolchain in order to streamline the
usage of the proposed methodology. The toolchain will be presented in
\sref{toolchain} whereas the methodology will be discussed in the next section.
