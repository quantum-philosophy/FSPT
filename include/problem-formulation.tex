Consider an in-design or in-use electronic system. The system is composed of a
number of processing elements and is capable of performing a number of
operations. The system receives a stream of requests or jobs, which it is
supposed to process. Each job implies a certain amount of work to be done and,
therefore, a certain amount of power/energy to be consumed and a certain amount
of heat to be dissipated. An example of such a system is a member of a computer
cluster.

Consider now a hypothetical research project targeted at developing a solution
of some sort, such as a management policy, for the system at hand. Since the
importance of power and temperature is well understood, which is motivated in
\sref{introduction}, the solution is required to take into account the two
quantities. Suppose further that the solution is to be based on learning from
power and temperature data.

Our goal is to develop a methodology for generating power and temperature traces
and a toolchain embodying it. The requirements are as follows:

\begin{itemize}
  \item {\bfseries Requirement~1 (Realism).}

  \item {\bfseries Requirement~2 (Speed).}
\end{itemize}
