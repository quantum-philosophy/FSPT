In this section, we describe the methodology that our toolchain follows. An
overview of the methodology is given in \fref{methodology} and can be summarized
as follows. There are two major stages: data acquisition and data synthesis. The
leftmost modules in \fref{methodology} correspond to the data-acquisition stage,
and the rightmost module to the data-synthesis stage. The data-acquisition stage
collects and stores reference data while the data-synthesis stage fetches these
data and produces power and temperature traces of the system. There are two
types of reference data: traffic and workload, which are referred to as
patterns. The two pieces characterize job arrivals and job workloads, and they
are to be discussed in \sref{traffic} and \sref{workload}, respectively, and to
be combined in \sref{composition}.

\subsection{Traffic} \slab{traffic}
As described in \sref{problem-formulation}, the system at hand serves a stream
of jobs, which can also be viewed as user requests. The foremost component of
our methodology is then a traffic model, describing how or, rather, when jobs
arrive. This model should satisfy a number of requirements in order to be
practically useful. First of all, the model should be able to capture well the
idiosyncrasies present in real traffic as it establishes a foundation for the
subsequent generation of power and temperature traces. Second, it should be
straightforward to configure the model given a dataset of arrival times serving
as a reference. This last requirement acknowledges the importance of the arrival
data that are at our disposal nowadays due to ubiquitously deployed monitoring
and logging systems.

The seminal work in \cite{leland1994} and subsequent studies have shown that
network traffic exhibits fractal properties such as burstiness, self-similarity,
and long-range dependence, which is very much unlike traditional telephone
traffic. It is well known that traffic models based on the Poisson process
\cite{lifshits2014} are unable to express any of the above properties and,
therefore, drastically departure from reality when it comes to modeling network
traffic. A step in the right direction is to consider the fractional Gaussian
noise \cite{lifshits2014}, which is a self-similar stochastic process. However,
the process is inconvenient to work with from the perspective of a computer
simulator. More concretely, the noise is suited for modeling arrivals per unit
of time but not arrival or inter-arrival times, which are what is typically
needed for simulation. Moreover, even with a proper rescaling and a positive
shift, the process can still take negative values, which is unrealistic.
Finally, the fractional Gaussian noise is a monofractal process; however, real
traffic data often have multifractal structures.

In order to address the aforementioned concerns and enable the generation of
arrival streams exhibiting fractal properties, our methodology employs the
multifractal wavelet model proposed in \cite{riedi1999}, characterizing
positive-valued data with long-range-dependent correlations. To elaborate, we
take a reference time series of arrival times, analyze it by means of the
discrete wavelet transform based on Haar wavelets, and construct a certain
representation of the data, which can then be used for generating random time
series matching the fractal properties of the original one. The model can be
tuned without any reference time series; however, the strength of the technique
is in the rigorous usage of reference data. By doing so, no manual parameter
tuning is needed, and the model gets tailored to the arrival pattern of each
particular problem.

To sum up, we now have a flexible technique for synthesizing arrival times,
which preserves the properties of real arrival data such as burstiness,
self-similarity, and long-range dependence.


\subsection{Workload} \slab{workload}
In the previous subsection, we introduced our approach to synthesizing arrival
times, capturing the properties of real arrival data such as burstiness,
self-similarity, and long-range dependence. Now we need to associate a concrete
workload with each arrival time or, equivalently, with each job or user request.
In this regard, there are two main aspects to discuss: the set of workload
candidates and the decision rule used to select a particular candidate for a
particular arrival.

Let us discuss workloads first. Keeping in mind the goal of this work, workloads
should satisfy a number of requirements. First, as emphasized throughout the
paper, we aim to produce realistic power and temperature traces; consequently,
the workloads should represent well the applications/services that the system is
supposed to provide to the end user. Second, a workload should be fast to
evaluate, which, in our context, refers to computing the power consumption of
that workload.

The particularities of the power consumption of a computer program are hard to
fabricate. A sequence of random numbers taken out of thin air will not do the
trick as programs have certain algorithmic structures. For instance, a program
might traverse a number of phases, and each phase might involve a number of
distinctive computations, shaping the corresponding power and temperature
profiles. Such features are important to preserve in order to make the
subsequent experimentation with machine-learning techniques and alike
meaningful.

With the above concern in mind, the workload-modeling part of our methodology is
founded on full-system simulations of representative programs. However, if we
had incorporated such simulations directly into our workflow, it would have
defeated the purpose of our work since, as motivated in \sref{introduction},
detailed simulations are too time consuming. Instead, we propose the use of
high-level recordings. To elaborate, we run each reference program under an
adequate simulator, capable of modeling the target platform, and record the
information that is needed for our data generation.\footnote{Such a technique is
similar in spirit to PinPlay \cite{patil2010}, which is a tool for recording and
replaying an execution of a program on the instruction level.} From our
experience, performance and power simulation is by far the largest expense on
the way to temperature, and, therefore, we propose to record is power directly,
eliminating this expense all together. The result of the above procedure is a
catalog of power traces corresponding to real programs, which we shall refer to
as power patterns.

Simulations obviously take time; however, they should be done only once.
Moreover, such a catalog of real power patterns can be populated and maintained
online by the research community so that the patterns are at a one-click
distance from any single researcher, and no expensive simulation is needed. The
role of the catalog could be similar to the one played by the benchmark suites
commonly used in research now-a-days, such as \sc{PARSEC} \cite{bienia2011} and
\sc{SPEC CPU2006} \cite{cpu2006}.

Now, it should be clear that, by recording power directly, we make a significant
trade-off: many details related to programs' executions have been discarded in
order to gain speed. What has been baked in into recordings cannot be to altered
at the usage/replay stage in general.

In our experiments, we use the applications from two benchmark suites, namely,
from \sc{PARSEC} and \sc{SPEC CPU2006}, which were mentioned earlier.


\subsection{Composition} \slab{composition}
As a result of the previous two subsections, we have obtained a stream of jobs.
This streams needs to be processed, which is the topic of this subsection. Here,
\emph{processing} refers to progressively building a schedule,\footnote{In this
paper, the mapping of tasks to processing elements is assumed to be a part of
the scheduling procedure.} constructing a power profile, and computing the
corresponding temperature profile. In \fref{methodology}, this functionality
resides in the box labeled ``Streamer.''

As motivated in \sref{motivation}, a prominent use case of our methodology is
the development of management strategies. Therefore, the management strategy
(including the scheduling policy) of the system at hand is devised by the user,
which is further detailed in \sref{usage-schemes}. Consequently, a scheduling
policy is given, and it decides on a schedule. Namely, for each arrived job, the
policy specifies when the job gets access to certain resources. Then we add
accordingly the recorded power pattern of the job to the power profile of the
system. A helpful analogy here might be paving a road with rectangular bricks of
different sizes. As the time goes by, we feed the power profile to a temperature
simulator and obtain a temperature profile.

The obtained power and temperature profiles are the final output of our
methodology. The key observation to be made is that no expensive performance or
power simulations are involved in the data-synthesis stage. The time consumed by
the procedure delineated above is practically negligible.

At this point, we would like to draw attention to the following. It should be
clear that, by recording power directly, we make a trade-off, and making this
trade-off is necessary. Many details pertaining to programs' executions have
been discarded in order to gain speed. What has been baked into recordings
cannot be to altered at the data-synthesis stage in general. Consider, for
instance, a recording of a program that had two cores at its disposal. This
pattern cannot be used to replay the execution as if there was only one core.
Similarly, the recording cannot tell what would happen if the program could
leverage one additional core. Another limitation concerns resource sharing,
which is twofold. First, workload patterns obtained in isolation cannot be used
to fabricate the interleaving of two programs (time sharing). Second, even if
two programs run on two different cores, they still can affect each other by
competing for such resources as shared L3 caches. Currently, the above concerns
can be addresses in our methodology only partially.


To summarize, the proposed methodology has been presented. The methodology is
composed of two stages. In the data-acquisition stage, reference traffic and
workload data are harvested. In the data-synthesis stage, the data are used to
generate a stream of jobs and subsequently compute a power and a temperature
profile. The produced data preserve the particularities of the reference data.
At the same time, the synthesis is fast as it bypasses expensive simulations.
