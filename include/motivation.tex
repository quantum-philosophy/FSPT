Accounting for power and temperature is key to achieving effectiveness,
efficiency, and robustness. However, power consumption and heat dissipation are
a result of a myriad of interactions between numerous hardware and software
components comprising modern electronic systems. It is, therefore, difficult to
describe mathematically or algorithmically the impact of a parameter on the
power and temperature characteristics of the system being developed. Moreover,
many aspects of the system are inherently uncertain to the designer. One of the
reasons is process variation \cite{chandrakasan2000}: the properties of
fabricated dies deviate from the nominal ones since process parameters cannot be
controlled precisely using the technologies currently employed in the
fabrication process. Another reason is aging \cite{coskun2006}: the performance
of a system degrades over time due to natural or accelerated wear, which is
unlikely to be uniform. Yet another and perhaps the most consequential reason is
workload: the demand on a system in the field is rarely, if ever, known in the
lab. Such uncertainties inevitably make power and temperature characteristics of
electronic systems uncertain as well.

The above concerns are particularly hard to address from a theoretical
standpoint. They can be better dealt with taking a more pragmatic approach,
namely, making use of on-chip data. To elaborate, assuming that the system at
hand provides basic profiling mechanisms such as performance counters and
thermal sensors, which is typically the case, direct or indirect power and
temperature readings are an invaluable source of information for making power-
and temperature-aware decisions. The attractiveness of the data route is due to
the fact that only one particular fabricated hardware in one particular
environment needs to be considered. More importantly, being on the chip enables
adaptation, which can also be made continuous. This means that instances of the
system are treated on a case-by-case basis; each instance attains a
custom-built, fine-tuned solution, and this solution evolves, diligently
following any changes.

Power and temperature data should be available timely. Proactive management
strategies are known to be more efficient than reactive ones since they aim at
preventing problems instead of recovering from the consequences of problems
\cite{coskun2008, chaudhry2015}, which, in particular, avoids slowing or
shutting down processing elements. Acting proactively requires peaking into the
future. Consequently, the first step towards proactive power- and
temperature-aware management---and, hence, towards making electronic products
more reliable and energy efficient---is to predict the power consumption and
heat dissipation.

Predicting the future relies on learning from the past, which is typically done
by virtue of tools from statistics and adjacent fields such as machine learning
\cite{bishop2006}. In \cite{coskun2008}, for example, temperature prediction is
based on an autoregressive moving-average model, enabling the development of an
efficient thermal management strategy for multiprocessor systems. The analysis
and mitigation of the impact of process variation in \cite{juan2014} are
facilitated by a linear-regression model trained on leakage measurements in
order to predict peak temperatures.

Regardless of the learning tool utilized, the tool needs data to learn from.
Prior to a physical instantiation of the platform under consideration, all the
hope rests on the shoulders of computer models and simulators. The need for
simulators is not any lower even when an existing platform is concerned. The
platform might not be available to the interested party, or it might not be an
appropriate place for early experimentation and fluid exploration, which is
arguably the most common scenario in research. In such cases, computer
simulators are of great help, and their ubiquitous usage speaks to this
assertion.

Simulations can be undertaken on different levels, depending on the goal in
mind. In order to be useful for learning purposes, simulations should be
sufficiently detailed so that they capture well the traits of real systems.
Although detailed simulations (down to cycle accuracy) are practical for the
design of individual components, such simulations fall short when it comes to
large systems. A modern multiprocessor system is reasonably complex, and it
might take days for a state-of-the-art simulator (not even cycle accurate) to
simulate a short, in wall-clock time, program running on such a system. This
scheme is not affordable for designing data-driven techniques as they require
many simulations with potentially large payloads.

To conclude, real data are rarely available, and simulation data are
prohibitively time consuming to obtain. Both are limiting, rigid from the
standpoint of a researcher trying to leverage the rich machinery of machine
learning. The need for alternative sources of high-quality data fuel is
prominent.
