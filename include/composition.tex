The stream of jobs needs to be now processed by progressively building a
schedule, constructing a power profile, and computing the corresponding
temperature profile. In \fref{methodology}, this functionality resides in the
box labeled ``Streamer.''

Since our infrastructure is to facilitate the development of intelligent
data-driven resource managers (see \sref{problem}), the manager of the system at
hand is devised by the user, which also includes the scheduling policy. For each
job, the policy specifies \emph{when} the job is granted computational
resources.

Once the job has been scheduled, we proceed to updating the power profile of the
system. The power profile is a matrix that specifies the power consumption of
the processing elements over discrete time moments; the rows and columns
traverse the processing elements and time moments, respectively. The workload
pattern of the job can also be seen as such a matrix but smaller. Then, in
accordance with the scheduling decision, each row of the workload pattern is
added to an appropriate row of the power profile starting from an appropriate
column.

There are several aspects that are worth noting. First, as mentioned in
\sref{workload}, we record dynamic and static power separately. We then make
sure that the static component is present even when no job is being
``executed.'' Second, the types of the processing elements that the workload
pattern was recorded on should be respected when scheduling and distributing the
pattern. Third, certain processing elements might be shared across several jobs,
which we discuss now.

The jobs that are mapped by the scheduling policy to disjoint sets of processing
elements require no special treatment. The interleaving of several jobs on a
core can be modeled by an appropriate interleaving of the corresponding workload
patterns: workload patterns have the notion of time, and they can be cut into
pieces and stitched back as needed. The joint effect of several jobs on a cache
can be reproduced by an additive model: the power values pertaining to the
cache, as captured by the workload patterns of the jobs in question, are summed
up. This approximation is sufficient as the power consumption of caches is
usually relatively low. Other effects originating from resource sharing are not
currently addressed in our methodology.

Finally, the temperature profile of the system is obtained by gradually feeding
the power profile into a temperature simulator. These two profiles are our final
output. They are fed into the resource manager as if they were the actual sensor
readings.
