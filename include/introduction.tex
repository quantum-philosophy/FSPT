\lettrine[findent=0.4em, nindent=0em]{\textbf{P}}{ower} consumption and heat
dissipation are of paramount importance. The two inseparable phenomena dictate
limitations on the usage of electronic devices and magnify the costs pertaining
to the deployment and maintenance of electronic systems. Power is essentially
energy, and energy translates willingly to hours of battery life and zeros in
electricity bills. Temperature, on the other hand, is one of the major causes of
permanent damage \cite{jedec}, which necessitates the deployment of adequate
cooling equipment, escalating the overall expenses. Furthermore, the situation
is deteriorated by the power-temperature interplay: higher power leads higher
temperature, and higher temperature strikes back by making electronic devices
consume even more power \cite{liu2007}. Under these circumstances, it is no
surprise that \emph{power}, \emph{temperature}, \emph{energy}, and their
derivatives are omnipresent in the titles of scientific publications. Power and
temperature have been steadily in the research limelight and have no plans on
leaving this spot.

Accounting for power and temperature is key to achieving effectiveness,
efficiency, and robustness. However, many aspects of the system being developed
are inherently uncertain to the designer. One of the reasons is process
variation \cite{chandrakasan2000}: the properties of fabricated dies deviate
from the nominal ones since process parameters cannot be controlled precisely
using the technologies currently employed in the fabrication process. Another
reason is aging \cite{coskun2006}: the performance of a system degrades over
time due to natural or accelerated wear, which is unlikely to be uniform. Yet
another and perhaps the most consequential reason is workload: the demand on a
system in the field is rarely, if ever, known in the lab. Such uncertainties
inevitably make power and temperature characteristics of electronic systems
uncertain as well.

The above concerns are especially prominent in offline settings. They can be
better dealt with online, assuming that the system at hand provides basic
profiling tools such as performance counters and thermal sensors, which is
typically the case. The attractiveness of the online route is due the fact that
only one particular fabricated hardware in one particular environment needs to
be considered. More importantly, being online enables continuous adaptation.
This means that instantiations of the system are treated on a case-by-case
basis; each instantiation attains a custom-built, fine-tuned solution, and this
solution evolves, diligently following changes. In this context, power and
temperature data are an invaluable source of information for making power- and
temperature-aware decisions.

Power and temperature data should be available timely. Proactive management
strategies are known to be more efficient than reactive ones since they aims at
preventing problems instead of recovering from the consequences of problems
\cite{coskun2008, beneventi2014}, which, in particular, avoids slowing or
shutting down processing elements. Acting proactively requires peaking into the
future. Consequently, the first step towards proactive power- and
temperature-aware management---and, hence, towards making electronic products
more reliable and energy efficient---is to predict the future power consumption
and heat dissipation.

Data-based prediction implies learning, which typically done by virtue of tools
from statistics and adjacent fields such as machine learning \cite{bishop2006}.
In \cite{coskun2008}, a thermal management strategy for multiprocessor systems
is proposed, in which temperature prediction is based on an autoregressive
moving-average model. The analysis and mitigation of the impact of process
variation are addressed in \cite{juan2014} by means of a liner-regression model
trained on leakage measurements in order to predict the maximal temperature. The
work in \cite{beneventi2014} presents a framework for learning thermal models of
multiprocessor chips from on-chip power and temperature measurements.

Regardless of the learning tool used, the tool needs data to learn from.

Real data.

From around 2006, artificial neural networks got a new spin.

Synthetic data.

Slow simulators.

Prior to a physical instantiation of a design, the hope is all in the hands of
computer models and simulators. The need in simulators is not any lower even
when an existing platform is concerned. The platform might not be available to
the interested party, or it might not be an appropriate place for early
experimentation and fluid exploration, which is arguably the most common
scenario in research. In any case, computer simulations are great help.

Simulation can be done on different levels. Cycle-accurate simulation is
particularly useful for the design of individual processing units; however,
cycle-accurate simulation falls short when it comes to large multiprocessor
systems. Such systems are reasonably more complex, which leads to prohibitively
large, often infeasible, simulation times. On the other hand, in order to be
properly addressed, many questions asked in both academia and industry do not
need cycle accuracy. It is important to have an adequate level of abstraction in
order to stay focused on what matters the most to the problem at hand without
being constantly destructed by insignificant or unrelated issues. In such cases,
cycle accuracy can become a serious obstacle.

Sniper raises the level of abstraction \cite{carlson2011}.

Ultra low-power inference \cite{park2015}.

Uncertainty.
