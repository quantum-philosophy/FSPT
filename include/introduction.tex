\lettrine[findent=0.4em, nindent=0em]{\textbf{P}}{ower} consumption and heat
dissipation are of paramount importance. The two inseparable phenomena dictate
limitations on the usage of electronic devices and magnify the costs pertaining
to the deployment and maintenance of electronic systems. Power is essentially
energy, and energy translates willingly to hours of battery life and zeros in
electricity bills. Temperature, on the other hand, is one of the major causes of
permanent damage \cite{jedec}, which necessitates the deployment of adequate
cooling equipment, escalating the overall expenses \cite{chaudhry2015}. The
situation is deteriorated even further by the power-temperature interplay:
higher power leads higher temperature, and higher temperature strikes back by
making electronic devices consume even more power \cite{liu2007}. Under these
circumstances, it is no surprise that \emph{power}, \emph{temperature},
\emph{energy}, and their derivatives are omnipresent in the titles of scientific
publications. Power and temperature have been steadily in the research limelight
and have no plans on leaving this spot.

In this paper, we set out to construct a methodology for assisting researchers
who develop power- and temperature-aware solutions that make use of
machine-learning techniques. Learning techniques obviously require data to learn
from, and these data might not be easily accessible to researchers, preventing
research ideas to flow freely.

That being said, it is important to note that, while synthetic data enable
thoughts and ideas to evolve and ripen, the solution being developed will
obviously have to face real data, and, therefore, it should be verified using
such data as well, which perhaps could be at the end of a large iteration.

The remainder of the paper is organized as follows. In the next section,
\sref{motivation}, we describe our primary motivation for embarking on this line
of research. Section~\ref{sec:prior-work} provides an overview of the prior
work. In \sref{present-work}, the contributions of the present work are
summarized. The problem that we address is formalized in
\sref{problem-formulation}. Our methodology and the corresponding toolchain are
described in \sref{methodology} and \sref{toolchain}, respectively. The
experimental results are reported and discussed in \sref{experimental-results}.
Section \ref{sec:conclusion} concludes the paper.
