\lettrine[findent=0.4em, nindent=0em]{\textbf{P}}{ower} consumption and heat
dissipation are of paramount importance. The two inseparable phenomena dictate
limitations on the usage of electronic devices and magnify the costs pertaining
to the deployment and maintenance of electronic systems. Power is essentially
energy, and energy translates willingly to hours of battery life and zeros in
electricity bills. Temperature, on the other hand, is one of the major causes of
permanent damage \cite{jedec}, which necessitates the deployment of adequate
cooling equipments, escalating the overall expenses \cite{chaudhry2015}. The
situation is deteriorated even further by the power-temperature interplay:
higher power leads to higher temperature, and higher temperature strikes back by
making devices consume even more power \cite{liu2007}. Under these
circumstances, it is no surprise that power and temperature have steadily been
in the research limelight and have no plans on leaving this spot.

In this paper, we set out to assist researchers working with power and
temperature in one specific but rather broad context. More concretely, we would
like to facilitate the development of on-chip, data-driven, power- and
temperature-aware solutions for multiprocessor systems. Let us now clarify the
terms involved in the previous sentence; the approach itself will be described
shortly after. \emph{On-chip} refers to taking decisions online or,
equivalently, at runtime, which is in contrast to having decisions taken
offline, at design time. \emph{Data-driven} refers to the usage of machine
learning \cite{bishop2006}, that is, to the usage of algorithms that learn from
data as opposed to algorithms of any other kind. \emph{Power-aware} refers to
decision-making that takes into account the impact on and the impact of power
(and energy) consumption. Similarly, \emph{temperature-aware} refers to
considering heat dissipation when taking decisions.

We approach the outlined objective by developing a methodology for a fast
synthesis of realistic power and temperature profiles. Our motivation for doing
so is described in the next section, \sref{motivation}, and here we would like
to remark that data-driven techniques obviously require data to learn from, and
these data might not be easily accessible to researchers for a variety of
reasons, preventing research ideas from flowing freely. The goal of this work is
to eliminate this obstacle.

It is important to realize right from the start that, while artificial data
enable thoughts and ideas to evolve and ripen, the solution being developed will
eventually have to face real data. Therefore, the solution should be tested and,
if needed, calibrated in a stage environment prior to the deployment to a
production environment, which could also be done periodically at the end of a
large development cycle. The usage of synthetic data substantially speeds up the
development process due to the flexibility and fast feedback enabled by such
data, which, in particular, means that one can filter out bad ideas at early
stages and focus solely on the ones which are viable.

The remainder of the paper is organized as follows. In the next section,
\sref{motivation}, we describe our motivation for embarking on this line of
research. Section~\ref{sec:prior-work} provides an overview of the prior work.
In \sref{present-work}, the contributions of the present work are summarized.
The problem that we address is formalized in \sref{problem-formulation}. The
proposed methodology and the corresponding toolchain are described in detail in
\sref{methodology} and \sref{toolchain}, respectively. The experimental results
are reported and discussed in \sref{experimental-results}. Section
\ref{sec:conclusion} concludes the paper.
