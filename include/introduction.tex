\lettrine[findent=0.3em, nindent=0em]{\textbf{P}}{ower consumption} and heat
dissipation are of great importance. Power is related to energy, and energy to
hours of battery life and to electricity bills. Temperature, on the other hand,
is one of the major causes of permanent damage \cite{jedec}, which necessitates
adequate cooling equipment escalating product expenses \cite{chaudhry2015}. The
situation is deteriorated further by the power-temperature interplay: higher
power leads to higher temperature, and higher temperature to higher power
\cite{liu2007}.

This work is to assist the researchers who work with power and temperature in
one specific but rather broad context: the development of data-driven power- and
temperature-aware solutions for multiprocessor systems; here \emph{data-driven}
refers to the usage of algorithms that learn from data \cite{bishop2006}. To
this end, a toolchain for fast synthesis of realistic power and temperature
profiles is developed. The primary motivation is that data-driven techniques
require data to learn from, and these data might not be easily accessible for a
variety of reasons, hampering research ideas. The goal of this work is to
eliminate this obstacle.

It is important to realize that the solution being developed will eventually
have to face real data and, therefore, should be tested and, if needed,
calibrated in a stage environment prior to the deployment to a production
environment. However, such synthetic data as the ones we generate can
substantially speed up this process due to the flexibility and fast feedback
that they provided. In particular, one can filter out inadequate ideas at early
stages and focus solely on those that are viable.

The remainder of the paper is organized as follows. The motivation behind this
work is given in \sref{motivation}. Section~\ref{sec:literature} provides an
overview of the prior work. In \sref{contribution}, our contributions are
summarized. The addressed problem is formalized in \sref{problem}. Our
methodology and toolchain are presented in \sref{methodology} and
\sref{toolchain}, respectively. The experimental results are given in
\sref{result}. Section \ref{sec:conclusion} concludes the paper.
