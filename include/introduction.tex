\lettrine[findent=0.4em, nindent=0em]{\textbf{P}}{power} consumption and heat
dissipation are of paramount importance. The two inseparable phenomena dictate
limitations on the usage of electronic devices and severely affect the costs
pertaining to the deployment and maintenance of electronic systems. Power is
essentially energy, and energy translates willingly to such quantities as hours
of battery life and zeros in electricity bills. Temperature, on the other hand,
is one of the major causes of permanent damage, and it necessitates the presence
of adequate cooling equipment, escalating the overall expenses. The situation is
deteriorated further by the fact that higher power leads higher temperature, and
higher temperature strikes back by causing electronic devices to consume even
more power. Under these circumstances, it is no surprise that \emph{power},
\emph{temperature}, \emph{energy}, and their derivatives are omnipresent in the
titles of scientific publications. Power and temperature have steadily been in
the research limelight and have no plans on leaving this spot.

Accounting for power and temperature, both at design time and runtime, is key to
achieving effectiveness, efficiency, and robustness. The ability to predict
power and temperature is then the first step towards this goal. Prior to a
physical instantiation of a design, the hope is all in the hands of computer
models and simulators. The need in simulators is not any lower even when an
existing platform is concerned. The platform might not be available to the
interested party, or it might not be an appropriate place for early
experimentation and fluid exploration, which is arguably the most common
scenario in research. In any case, computer simulations are great help.

Ultra low-power inference \cite{park2015}.

Uncertainty.

Machine learning.

Neural networks.

Training data.

Real data.

Synthetic data.

Slow simulators.

Cycle-accurate simulation is an important design tool. It is particularly useful
for the design of individual processing units; however, cycle-accurate
simulation falls short when it comes to large multiprocessor systems. Such
systems are reasonably more complex, which leads to prohibitively large, often
infeasible, simulation times. On the other hand, in order to be properly
addressed, many questions asked in both academia and industry do not need cycle
accuracy. It is important to have an adequate level of abstraction in order to
stay focused on what matters the most to the problem at hand without being
constantly destructed by insignificant or unrelated issues. In such cases, cycle
accuracy can become a serious obstacle.

Sniper raises the level of abstraction \cite{carlson2011}.
