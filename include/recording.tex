\begin{table}
  \begin{threeparttable}
    \caption{Recording of the \sc{PARSEC} benchmark suite}
    \begin{tabular*}{\linewidth}{=L{43pt}-R{24pt}-R{24pt}-R{24pt}-R{17pt}-R{17pt}-R{17pt}}
      \toprule
      & \multicolumn{3}{c}{Recording time, m} & \multicolumn{3}{c}{Simulated time, s} \\
      Program       & S & M & L & S & M & L \\
      \cmidrule( r){1-1}
      \cmidrule(l ){2-4}
      \cmidrule(l ){5-7}
      blackscholes  &  10.77 &  44.21 & 0 & 0.08 & 0.31 & 0 \\
      bodytrack     &  30.58 & 120.04 & 0 & 0.19 & 0.80 & 0 \\
      canneal       &  36.90 & 104.52 & 0 & 0.17 & 0.63 & 0 \\
      dedup         &  81.26 & 208.80 & 0 & 1.66 & 2.86 & 0 \\
      facesim       & 760.59 & 790.80 & 0 & 8.54 & 8.47 & 0 \\
      ferret        &  49.67 & 159.19 & 0 & 0.49 & 1.32 & 0 \\
      fluidanimate  &  46.55 & 131.00 & 0 & 0.59 & 1.43 & 0 \\
      freqmine      &  74.45 & 292.28 & 0 & 0.72 & 2.75 & 0 \\
      raytrace      & 253.23 & 341.32 & 0 & 0.28 & 0.75 & 0 \\
      streamcluster &  44.57 & 172.50 & 0 & 0.36 & 1.45 & 0 \\
      swaptions     &  35.20 & 140.02 & 0 & 0.24 & 0.97 & 0 \\
      vips          &  94.00 & 293.51 & 0 & 0.57 & 1.71 & 0 \\
      x264          &  29.36 & 155.16 & 0 & 0.19 & 1.05 & 0 \\
      \bottomrule
    \end{tabular*}
    \tlab{recording}
    \begin{tablenotes}
      \item Note the difference in the units of measurement: the recording time
      is in minutes, and the simulated time is in seconds. S, M, and L stand
      for ``small,'' ``medium,'' and ``large,'' respectively, and signify input
      size.
    \end{tablenotes}
  \end{threeparttable}
\end{table}
% vim: nowrap tw=0

In the above, we outlined how the reference workload data were harvested using
the Recorder tool and the infrastructure around it, which were discussed in
\sref{recorder} and are displayed in \fref{recorder}. Now we show and elaborate
on the performance characteristics of this recording process.

The benchmark suite that we shall look at is \sc{PARSEC}. Our findings are
summarized in \tref{recording}. \sc{PARSEC} provides several choices of inputs
to the programs, and we recorded each program with three different inputs,
namely, with the ones classified as small, medium, and large. There are two
types of information shown in \tref{recording}: recording time (in hours), which
is the time that was taken to simulate and record the programs, and simulated
time (in seconds), which is the time that the programs would have taken in real
life. The sampling interval used in all experiments is one millisecond.

Each input class was recorded in a single batch: all 13 programs were simulated
simultaneously using 13 Sniper processes, which was explained and motivated in
\sref{streamer}. Consequently, the total recording time with respect to each
batch is dictated by the program that took the most time to finish. For small
and medium inputs, this program was \texttt{facesim}, which took approximately
13 hours in both cases. The simulated times of \texttt{facesim} indicate that
\sc{PARSEC} actually has only one input size for this particular program.
Regarding large inputs, \texttt{freqmine} finished last; more concretely, the
program took 18 hours. As an aside for the curious reader, the simulated and
recording times of \sc{SPEC CPU2006} (not shown) were an order of magnitude
larger than the ones of \sc{PARSEC}.

It can be seen in \tref{recording} that the throughput in terms of simulated
time is (expectedly) low: roughly speaking, two--three hours of recording time
amounts to one second of simulated time. However, it is important to realize
that these are one-time expenses in our methodology; the situation would be
drastically worse if one had to perform such simulations all the time (see
\sref{workload}). Another important aspect to note is that the observed
recording times have been substantially reduced by the choice of performance
simulator---Sniper is based on novel simulation ideas \cite{carlson2011}---and
the parallelization strategy and caching mechanism described in \sref{streamer}.
