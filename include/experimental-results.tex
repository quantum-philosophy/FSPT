In this section, we illustrate the performance of the toolchain presented in
\sref{toolchain}. All experiments are conducted on a \sc{GNU}/Linux machine
equipped with 16 processors Intel Xeon E5520 2.27~\sc{GH}z and 24~\sc{GB} of
\sc{RAM}.

To begin with, we would like to describe how we approached acquiring reference
traffic and workload data for our experiments; see \sref{traffic} and
\sref{workload}, respectively. This information might also serve as a set of
guidelines for potential users of the proposed methodology and toolbox.

In order to obtain real-life arrival patterns, we used a dataset published by
Google \cite{google}. The dataset contains usage data of a computer cluster over
a month period, namely, over May 2011. We downloaded the table that was tracking
the life cycles of the jobs submitted to the cluster and extracted the time
stamps of the first event related to each job. As a result, we obtained around
670,000 data points, which we used for model fitting as it was described in
\sref{traffic}. The described processing strategy is rather straightforward and
could be a good place to start. However, one can query the data in more
sophisticated ways in order to distill more specific and accurate information
(for instance, to focus on a subset of the cluster's nodes).

\begin{table}
  \caption{Target architecture}
  \begin{tabular*}{\linewidth}{=L{70pt}l}
    \toprule
    Component    & Description \\
    \midrule
    Core         & 2660 MHz, 1.2 V \\
    L1-I/D cache & 32 KB, 4-way, LRU, private \\
    L2 cache     & 256 KB, 4-way, LRU, private \\
    L3 cache     & 8192 KB, 16-way, LRU, one per four cores \\
    \bottomrule
  \end{tabular*}
  \tlab{target}
\end{table}
% vim: nowrap tw=0

Regarding workload patters, we decided to use the benchmark suites that are
commonly utilized in research nowadays. With our toolbox in place, this approach
is even natural as Sniper provides a smooth integration with some of the most
popular benchmark suites out of the box. In our experiments, the workload
patterns were obtained by simulating and recording (via our recording
infrastructure displayed in \fref{recorder}) the programs from the popular
\sc{PARSEC} \cite{bienia2011} and \sc{SPEC CPU2006} \cite{cpu2006} benchmark
suites; the former contains 13 programs, and the latter 29 programs. The
architecture used in these simulations is outlined in \tref{target}, which
corresponds to Intel's Nehalem-based Gainestown series. Many programs can be
executed with inputs of different sizes. Sniper also makes it easy to work with
arbitrary programs and experiment with different x86-based architecture setups.

All reference data that we collected and processed to make them suitable for our
toolbox are available online \cite{sources}.

\subsection{Data Acquisition}
\begin{table}
  \begin{threeparttable}
    \caption{Recording of the \sc{PARSEC} benchmark suite}
    \begin{tabular*}{\linewidth}{=L{43pt}-R{24pt}-R{24pt}-R{24pt}-R{17pt}-R{17pt}-R{17pt}}
      \toprule
      & \multicolumn{3}{c}{Recording time, m} & \multicolumn{3}{c}{Simulated time, s} \\
      Program       & S & M & L & S & M & L \\
      \cmidrule( r){1-1}
      \cmidrule(l ){2-4}
      \cmidrule(l ){5-7}
      blackscholes  &  10.77 &  44.21 & 0 & 0.08 & 0.31 & 0 \\
      bodytrack     &  30.58 & 120.04 & 0 & 0.19 & 0.80 & 0 \\
      canneal       &  36.90 & 104.52 & 0 & 0.17 & 0.63 & 0 \\
      dedup         &  81.26 & 208.80 & 0 & 1.66 & 2.86 & 0 \\
      facesim       & 760.59 & 790.80 & 0 & 8.54 & 8.47 & 0 \\
      ferret        &  49.67 & 159.19 & 0 & 0.49 & 1.32 & 0 \\
      fluidanimate  &  46.55 & 131.00 & 0 & 0.59 & 1.43 & 0 \\
      freqmine      &  74.45 & 292.28 & 0 & 0.72 & 2.75 & 0 \\
      raytrace      & 253.23 & 341.32 & 0 & 0.28 & 0.75 & 0 \\
      streamcluster &  44.57 & 172.50 & 0 & 0.36 & 1.45 & 0 \\
      swaptions     &  35.20 & 140.02 & 0 & 0.24 & 0.97 & 0 \\
      vips          &  94.00 & 293.51 & 0 & 0.57 & 1.71 & 0 \\
      x264          &  29.36 & 155.16 & 0 & 0.19 & 1.05 & 0 \\
      \bottomrule
    \end{tabular*}
    \tlab{recording}
    \begin{tablenotes}
      \item Note the difference in the units of measurement: the recording time
      is in minutes, and the simulated time is in seconds. S, M, and L stand
      for ``small,'' ``medium,'' and ``large,'' respectively, and signify input
      size.
    \end{tablenotes}
  \end{threeparttable}
\end{table}
% vim: nowrap tw=0

In the above, we described in a nutshell how the reference workload data were
harvested using our Recorder tool and the infrastructure around it, which were
discussed in \sref{recorder} and are displayed in \fref{recorder}. Now we show
and elaborate on the performance characteristics of this recording process.

The benchmark suite that we shall look at is \sc{PARSEC}, and our findings are
summarized in \tref{recording}. \sc{PARSEC} provides several choices of inputs
to the programs, and we recorded each program with three different inputs,
namely, with the ones classified as small, medium, and large. There are two
types of information shown in \tref{recording}: recording time (in minutes),
which is the time that was taken to simulate and record the programs, and
simulated time (in seconds), which is the time that the programs would have
taken in real life.

Each input class was recorded in a single batch: all 13 programs were simulated
simultaneously using 13 Sniper processes, which was explained and motivated in
\sref{streamer}. Consequently, the total recording time with respect to each
batch is dictated by the program that took the most time to finish. This program
was \texttt{facesim} in all the cases. It took approximately 13 hours to
simulate \texttt{facesim}, and this time was observed, again, in all the cases,
which indicates that \sc{PARSEC} actually has only one input size for this
particular program. As an aside for the curious reader, the simulated and,
hence, recording times of \sc{SPEC CPU2006} (not shown) were an order of
magnitude larger than the ones of \sc{PARSEC}.

It can be seen in \tref{recording} that the throughput in terms of simulated
time is extremely low: one week of recording time gives roughly one second of
simulated time. This throughput is expected and affordable as a one-time cost.
However, it is severely onerous for experimenting with data-driven techniques
wherein lots of data are typically needed, and these data might need to be
recalculated each time a parameter changes.

Let us remark that the affordability of simulations is a merit of the ideas
underpinning the Sniper simulator \cite{carlson2011}, which our recording
infrastructure is based on (see \fref{recorder}). It is also important to note
that the recording times reported earlier would have been substantially larger
without the parallelization and caching mechanism described in \sref{streamer}.

\subsection{Data Synthesis}
