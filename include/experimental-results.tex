This section illustrates the performance of the toolchain presented in
\sref{toolchain}. All experiments are conducted on a \sc{GNU}/Linux machine
equipped with 16 processors Intel Xeon E5520 2.27~\sc{GH}z and 24~\sc{GB} of
\sc{RAM}.

To begin with, we would like to describe how we approached acquiring reference
traffic and workload data for our experiments; see \sref{traffic} and
\sref{workload}, respectively. This information might serve as a set of
guidelines for potential users of the proposed methodology and toolbox.

In order to obtain real-life traffic patterns, we used a dataset published by
Google \cite{google}. The dataset contains usage data of a computer cluster over
a month period, namely, over May 2011. We downloaded the table tracking the life
cycles of the jobs submitted to the cluster and extracted the time stamps of the
first event related to each job. As a result, we obtained around 670,000 data
points, which we used for model fitting as it was described in \sref{traffic}.
The processing strategy described here is rather straightforward and could be a
good place to start. However, one can query the data in more sophisticated ways
in order to distill more specific and accurate information (for instance, to
focus on a subset of the cluster's nodes).

\begin{table}
  \caption{Target architecture}
  \begin{tabular*}{\linewidth}{=L{70pt}l}
    \toprule
    Component    & Description \\
    \midrule
    Core         & 2660 MHz, 1.2 V \\
    L1-I/D cache & 32 KB, 4-way, LRU, private \\
    L2 cache     & 256 KB, 4-way, LRU, private \\
    L3 cache     & 8192 KB, 16-way, LRU, one per four cores \\
    \bottomrule
  \end{tabular*}
  \tlab{target}
\end{table}
% vim: nowrap tw=0

Regarding workload patters, we made use of the benchmark suites that are
commonly utilized in research nowadays. With our toolbox in place, this approach
is even natural as Sniper provides a smooth integration with some of the most
popular benchmark suites out of the box. In our experiments, the workload
patterns were obtained by simulating and recording (via our recording
infrastructure displayed in \fref{recorder}) the programs from the popular
\sc{PARSEC} \cite{bienia2011} and \sc{SPEC CPU2006} \cite{cpu2006} benchmark
suites; the former contains 13 programs, and the latter 29 programs. The
architecture used in these simulations is outlined in \tref{target}, which
corresponds to Intel's Nehalem-based Gainestown series. Many programs can be
executed with inputs of different sizes. Sniper also makes it easy to work with
arbitrary programs and experiment with different x86-based architecture setups.

All reference data that we collected and processed to make them suitable for our
toolbox are available online \cite{sources}.

\subsection{Recording}
\begin{table}
  \begin{threeparttable}
    \caption{Recording of the \sc{PARSEC} benchmark suite}
    \begin{tabular*}{\linewidth}{=L{43pt}-R{24pt}-R{24pt}-R{24pt}-R{17pt}-R{17pt}-R{17pt}}
      \toprule
      & \multicolumn{3}{c}{Recording time, m} & \multicolumn{3}{c}{Simulated time, s} \\
      Program       & S & M & L & S & M & L \\
      \cmidrule( r){1-1}
      \cmidrule(l ){2-4}
      \cmidrule(l ){5-7}
      blackscholes  &  10.77 &  44.21 & 0 & 0.08 & 0.31 & 0 \\
      bodytrack     &  30.58 & 120.04 & 0 & 0.19 & 0.80 & 0 \\
      canneal       &  36.90 & 104.52 & 0 & 0.17 & 0.63 & 0 \\
      dedup         &  81.26 & 208.80 & 0 & 1.66 & 2.86 & 0 \\
      facesim       & 760.59 & 790.80 & 0 & 8.54 & 8.47 & 0 \\
      ferret        &  49.67 & 159.19 & 0 & 0.49 & 1.32 & 0 \\
      fluidanimate  &  46.55 & 131.00 & 0 & 0.59 & 1.43 & 0 \\
      freqmine      &  74.45 & 292.28 & 0 & 0.72 & 2.75 & 0 \\
      raytrace      & 253.23 & 341.32 & 0 & 0.28 & 0.75 & 0 \\
      streamcluster &  44.57 & 172.50 & 0 & 0.36 & 1.45 & 0 \\
      swaptions     &  35.20 & 140.02 & 0 & 0.24 & 0.97 & 0 \\
      vips          &  94.00 & 293.51 & 0 & 0.57 & 1.71 & 0 \\
      x264          &  29.36 & 155.16 & 0 & 0.19 & 1.05 & 0 \\
      \bottomrule
    \end{tabular*}
    \tlab{recording}
    \begin{tablenotes}
      \item Note the difference in the units of measurement: the recording time
      is in minutes, and the simulated time is in seconds. S, M, and L stand
      for ``small,'' ``medium,'' and ``large,'' respectively, and signify input
      size.
    \end{tablenotes}
  \end{threeparttable}
\end{table}
% vim: nowrap tw=0

In the above, we outlined how the reference workload data were harvested using
Recorder and the infrastructure around it (recall \sref{recorder} and
\fref{recorder}). Let us now elaborate on the performance characteristics of
that recording process.

The benchmark suite that we shall look at is \sc{PARSEC}. Our findings are
summarized in \tref{recording}. \sc{PARSEC} provides several choices of inputs
to the programs, and each program was recorded with three different inputs,
namely, with the ones classified as small, medium, and large. There are two
types of information shown in \tref{recording}: recording time (in hours), which
is the time that was taken to simulate and record the programs, and simulated
time (in seconds), which is the time that the programs would have taken in real
life. The sampling interval used in all the experiments was one millisecond.

Each input class was recorded in a single batch: all 13 programs were simulated
at the same time using 13 Sniper processes, which is explained and motivated in
\sref{streamer}. Consequently, the total recording time with respect to each
batch is dictated by the program that took the most time to finish. For small
and medium inputs, this program was \texttt{facesim}, which took approximately
13 hours in both cases. The simulated times of \texttt{facesim} indicate that
\sc{PARSEC} actually has only one input size for this particular program.
Regarding large inputs, \texttt{freqmine} finished last; more concretely, the
program took 18 hours. As an aside for the interested reader, the simulated and
recording times of \sc{SPEC CPU2006} (not shown) were an order of magnitude
larger than the ones of \sc{PARSEC}.

It can be seen in \tref{recording} that the throughput in terms of simulated
time is (expectedly) low: roughly speaking, two--three hours of recording time
amounts to one second of simulated time. However, it is important to realize
that these are one-time expenses in our methodology; the situation would be much
worse if one had to perform such simulations all the time (see \sref{workload}).
Another important aspect to note is that the observed recording times have been
substantially reduced by the choice of performance simulator---Sniper is based
on novel simulation ideas \cite{carlson2011}---and the parallelization strategy
and caching mechanism described in \sref{streamer}.


\subsection{Streaming}
\setlength{\tabcolsep}{4pt}
\begin{table}
  \begin{threeparttable}
    \caption{Streaming (synthesis of power and temperature profiles)}
    \begin{tabular*}{\linewidth}{=L{50pt}-R{40pt}-R{40pt}-R{40pt}-R{40pt}}
      \toprule
      \multirow{2}{*}{\parbox{50pt}{Synthesized time (seconds)}} & \multicolumn{4}{c}{Synthesis time (seconds)} \\
      & $4 + 1$ & $8 + 2$ & $16 + 4$ & $32 + 8$ \\
      \cmidrule( r){1-1}
      \cmidrule(l ){2-5}
         10 &   0.24 &   0.40 &   0.67 &   1.13 \\
        100 &   2.11 &   3.66 &   6.18 &  10.19 \\
       1000 &  20.59 &  37.47 &  64.58 & 104.00 \\
      10000 & 214.98 & 394.84 & 598.89 & 984.59 \\
      \bottomrule
    \end{tabular*}
    \tlab{streaming}
    \begin{tablenotes}
      \item \hspace{-0.70em}``$M + N$'' stands for $M$ cores and $N$ L3 caches.
      Every four cores have a shared L3 cache; therefore, it holds that
      $N = M / 4$.
    \end{tablenotes}
  \end{threeparttable}
\end{table}
% vim: nowrap tw=0

Let us turn to Streamer, which corresponds to the data-synthesis stage (recall
\sref{streamer} and \fref{streamer}). Our objective in this subsection is to
study the scalability of the tool as measured by synthesis time, which is the
time that is needed for the tool to synthesize power and temperature profiles
under certain conditions or requirements. In these experiments, workload
patterns are assigned to job arrivals randomly, and a simple
first-in-first-served scheduling policy is assumed; both are the default but
easily replaceable options used by the toolchain.

The preformed experiments are consolidated in \tref{streaming}. We report our
synthesis time along two axes: synthesized time (rows) and platform size
(columns). The former is analogous to simulated time, and the latter represents
different platforms as follows. Each considered platform is composed of a number
of cores, and there is a single L3 cache for every four cores; both cores and
caches are referred to as processing elements. Platform size is defined as the
number of processing elements, and it is denoted by ``$M + N$'' in
\tref{streaming} where $M$ and $N$ are the numbers of cores and L3 caches,
respectively.

Unlike the throughput of simulation (discussed in the previous subsection), the
throughput of synthesis in terms of synthesized time is very high, which is well
supported by \tref{streaming}. To give an example, it takes Streamer around a
minute to produce power and temperature data that are worth around 17 minutes of
runtime of a computer system with 16 cores, which would be practically
infeasible to achieve with full-fledged simulations (refer to the results given
in \tref{recording}).

Another observation made from \tref{streaming} is that synthesis time scales
linearly with respect to the length of the time span being synthesized
(synthesized time), and the same can be concluded regarding the other dimension,
platform size. The growth with respect to platform size is due to the increasing
complexity of the underlying thermal \sc{RC} circuit used for temperature
simulation; thermal circuits are elaborated on in \sref{streamer}.


To summarize, we have shown and discussed the performance of the Recorder and
Streamer tools.
