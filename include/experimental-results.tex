\begin{table}
  \begin{threeparttable}
    \caption{Target architecture}
    \begin{tabular*}{\linewidth}{=L{70pt}l}
      \toprule
      Component    & Description \\
      \midrule
      Core         & 2660 MHz, 1.2 V \\
      L1-I/D cache & 32 KB, 4-way, LRU, private \\
      L2 cache     & 256 KB, 4-way, LRU, private \\
      L3 cache     & 8192 KB, 16-way, LRU, one per four cores \\
      \bottomrule
    \end{tabular*}
    \tlab{target}
    \begin{tablenotes}
      \item A detailed description of the target architecture can be found in
      the \texttt{nehalem.cfg} and \texttt{gainestown.cfg} configuration files
      of Sniper.
    \end{tablenotes}
  \end{threeparttable}
\end{table}
% vim: nowrap tw=0

This section illustrates the performance of the toolchain presented in
\sref{toolchain}. All experiments are conducted on a \sc{GNU}/Linux machine
equipped with 16 processors Intel Xeon E5520 2.27~\sc{GH}z and 24~\sc{GB} of
\sc{RAM}.

To begin with, we would like to describe how we acquired reference traffic and
workload data for our experiments; the two were discussed in \sref{traffic} and
\sref{workload}, respectively. This information might serve as a set of
guidelines for potential users of the proposed methodology and toolbox.

In order to obtain real-life traffic patterns, we used a dataset published by
Google \cite{google}. The dataset contains usage data of a computer cluster over
a month period, namely, over May 2011. We downloaded the table tracking the life
cycles of the jobs submitted to the cluster and extracted the time stamps of the
first event related to each job. As a result, we obtained around 670,000 data
points, which we used for model fitting as it was described in \sref{traffic}.
Needless to mention that one can query the data in other ways in order to
distill traffic patterns that are more relevant to particular problems.

Regarding workload patters, we made use of the benchmark suites that are
commonly utilized in research nowadays. With our toolbox in place, this approach
is even natural as Sniper provides a smooth integration with some of the most
popular benchmark suites out of the box. In our experiments, the workload
patterns were obtained by simulating and recording (via our recording
infrastructure displayed in \fref{recorder}) the programs from the popular
\sc{PARSEC} \cite{bienia2011} and \sc{SPEC CPU2006} \cite{cpu2006} benchmark
suites; the former contains 13 programs, and the latter 29 programs. The
architecture used in these simulations is outlined in \tref{target}, which
corresponds to Intel's Nehalem-based Gainestown series. Many programs can be
executed with inputs of different sizes. Sniper also makes it easy to work with
arbitrary programs and experiment with different x86-based architecture setups.

All reference data that we collected and processed in order to make them
suitable for our toolbox are available online at \cite{sources}, which also
contains a detailed description of the target architecture used for the
recording of workload patterns.

\subsection{Recording}
\setlength{\tabcolsep}{4pt}
\begin{table}
  \begin{threeparttable}
    \caption{Recording (the \sc{PARSEC} benchmark suite)}
    \begin{tabular*}{\linewidth}{=L{43pt}-R{24pt}-R{28pt}-R{24pt}-R{24pt}-R{28pt}-R{24pt}}
      \toprule
      & \multicolumn{3}{c}{Recording time (hours)} & \multicolumn{3}{c}{Simulated time (seconds)} \\
      Program       & Small & Medium & Large & Small & Medium & Large \\
      \cmidrule( r){1-1}
      \cmidrule(l ){2-4}
      \cmidrule(l ){5-7}
      blackscholes  &  0.18 &  0.74 &  3.00 & 0.07 & 0.28 &  1.13 \\
      bodytrack     &  0.51 &  2.00 &  7.36 & 0.17 & 0.70 &  2.71 \\
      canneal       &  0.61 &  1.74 &  4.04 & 0.18 & 0.67 &  1.72 \\
      dedup         &  1.35 &  3.48 & 17.20 & 1.97 & 2.85 & 10.75 \\
      facesim       & 12.68 & 13.18 & 15.48 & 7.84 & 7.93 &  7.87 \\
      ferret        &  0.83 &  2.65 & 12.15 & 0.40 & 1.06 &  4.59 \\
      fluidanimate  &  0.78 &  2.18 &  6.90 & 0.53 & 1.32 &  4.11 \\
      freqmine      &  1.24 &  4.87 & 18.04 & 0.67 & 2.63 &  9.21 \\
      raytrace      &  4.22 &  5.69 & 10.08 & 0.24 & 0.63 &  1.51 \\
      streamcluster &  0.74 &  2.88 & 15.71 & 0.34 & 1.40 & 10.68 \\
      swaptions     &  0.59 &  2.33 &  9.05 & 0.23 & 0.91 &  3.64 \\
      vips          &  1.57 &  4.89 & 14.47 & 0.53 & 1.59 &  4.39 \\
      x264          &  0.49 &  2.59 &  8.61 & 0.17 & 0.99 &  3.09 \\
      \bottomrule
    \end{tabular*}
    \tlab{recording}
    \begin{tablenotes}
      \item \emph{Small}, \emph{medium}, and \emph{large} signify the input size
      of the programs.
    \end{tablenotes}
  \end{threeparttable}
\end{table}
% vim: nowrap tw=0

In the above, we outlined how the reference workload data were harvested using
the Recorder tool and the infrastructure around it, which were discussed in
\sref{recorder} and are displayed in \fref{recorder}. Now we show and elaborate
on the performance characteristics of this recording process.

The benchmark suite that we shall look at is \sc{PARSEC}. Our findings are
summarized in \tref{recording}. \sc{PARSEC} provides several choices of inputs
to the programs, and we recorded each program with three different inputs,
namely, with the ones classified as small, medium, and large. There are two
types of information shown in \tref{recording}: recording time (in hours), which
is the time that was taken to simulate and record the programs, and simulated
time (in seconds), which is the time that the programs would have taken in real
life. The sampling interval used in all experiments is one millisecond.

Each input class was recorded in a single batch: all 13 programs were simulated
simultaneously using 13 Sniper processes, which was explained and motivated in
\sref{streamer}. Consequently, the total recording time with respect to each
batch is dictated by the program that took the most time to finish. For small
and medium inputs, this program was \texttt{facesim}, which took approximately
13 hours in both cases. The simulated times of \texttt{facesim} indicate that
\sc{PARSEC} actually has only one input size for this particular program.
Regarding large inputs, \texttt{freqmine} finished last; more concretely, the
program took 18 hours. As an aside for the curious reader, the simulated and
recording times of \sc{SPEC CPU2006} (not shown) were an order of magnitude
larger than the ones of \sc{PARSEC}.

It can be seen in \tref{recording} that the throughput in terms of simulated
time is (expectedly) low: roughly speaking, two--three hours of recording time
amounts to one second of simulated time. However, it is important to realize
that these are one-time expenses in our methodology; the situation would be
drastically worse if one had to perform such simulations all the time (see
\sref{workload}). Another important aspect to note is that the observed
recording times have been substantially reduced by the choice of performance
simulator---Sniper is based on novel simulation ideas \cite{carlson2011}---and
the parallelization strategy and caching mechanism described in \sref{streamer}.


\subsection{Streaming}
\setlength{\tabcolsep}{4pt}
\begin{table}
  \begin{threeparttable}
    \caption{Streaming}
    \begin{tabular*}{\linewidth}{=L{50pt}-R{40pt}-R{40pt}-R{40pt}-R{40pt}}
      \toprule
      \multirow{2}{*}{\parbox{50pt}{Synthesized time (seconds)}} & \multicolumn{4}{c}{Synthesis time (seconds)} \\
      & $4 + 1$ & $8 + 2$ & $16 + 4$ & $32 + 8$ \\
      \cmidrule( r){1-1}
      \cmidrule(l ){2-5}
         10 &   0.24 &   0.40 &   0.67 &   1.13 \\
        100 &   2.11 &   3.66 &   6.18 &  10.19 \\
       1000 &  20.59 &  37.47 &  64.58 & 104.00 \\
      10000 & 214.98 & 394.84 & 598.89 & 984.59 \\
      \bottomrule
    \end{tabular*}
    \tlab{streaming}
    \begin{tablenotes}
      \item ``$M + N$'' stands for $M$ cores and $N$ L3 caches. Each four cores
        have a shared L3 cache; therefore, it holds that $N = M / 4$.
    \end{tablenotes}
  \end{threeparttable}
\end{table}
% vim: nowrap tw=0

Let us turn to the Streaming tool, which corresponds to the data-synthesis stage
of the methodology (see \fref{methodology}). Streamer was introduced in
\sref{streamer} and is depicted in \fref{streamer}. Our objective in this
subsection is to study the scalability of the tool as measured by synthesis
time, which is the time that is needed for the tool to synthesize power and
temperature profiles under certain conditions or requirements.

Before we proceed, let us reiterate that, as far as Streamer is concerned, the
expense of performance and power simulation is essentially zero as the tool
works with precomputed power data (power patterns). The most time-consuming part
of Streamer is temperature simulation, which is actually negligibly small (see
below) compared to the time shown in \tref{recording}.

The preformed experiments are consolidated in \tref{streaming}. We report
synthesis time along two dimensions: synthesized time (rows) and platform size
(columns). The former is analogous to simulated time, and the latter represents
different platforms as follows. Each considered platform is composed of a number
of cores, and there is an L3 cache for every four cores; both cores and caches
are referred to as processing elements. Platform size is then defined as the
number of processing elements, and it is denoted by ``$M + N$'' in
\tref{streaming} where $M$ and $N$ are the numbers of cores and caches,
respectively.\footnote{The specifications of the considered platforms are a part
of our supplementary materials available at \cite{sources}.}

Unlike the throughput of simulation (discussed in the previous subsection), the
throughput of synthesis in terms of synthesized time is very high, which is well
supported by \tref{streaming}. For instance, it takes Streamer around a minute
to produce power and temperature data that are worth around 17 minutes of
runtime of a multiprocessor system with 16 cores, which would be practically
infeasible to achieve with full-fledged simulations (refer to \tref{recording}).

Another observation made from \tref{streaming} is that synthesis time scales
linearly with respect to the length of the time span being synthesized
(synthesized time), and the same can be concluded regarding platform size. The
growth with respect to platform size is due to the increasing complexity of the
underlaying thermal \sc{RC} circuit used for temperature simulation; thermal
circuits were elaborated on in \sref{streamer}.


To summarize, we have discussed the performance of Recorder and Streamer. The
results reported in \tref{recording} motivate our work and communicate well the
message of this paper: the speed of the state-of-the-art simulators is severely
onerous for the purpose of experimenting with data-driven techniques. The core
problem is that such techniques typically require lots of data (long execution
traces); moreover, these data might need to be recalculated each time a
parameter changes (for instance, the parameterization of the scheduling policy).
The results in \tref{streaming} show that the proposed approach can efficiently
tackle this problem by taking the data burden away and, hence, making it easier
to experiment with data-driven techniques.
