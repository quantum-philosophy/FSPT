The present work is related to computer simulation. For our purposes, it is
sufficient to distinguish four types of simulation: traffic, performance, power,
and temperature, which are shown in \fref{development}. In this paper,
\emph{traffic} refers to a stream of jobs that lade the system with work, and
\emph{performance} to various performance metrics of the system such as the
numbers of executed instructions and read/write memory accesses.

Let us discuss traffic first. The Poisson process \cite{lifshits2014} is
arguably the most well-known model in this regard. However, the seminal work in
\cite{leland1994} and subsequent studies have shown that network traffic
exhibits fractal properties such as burstiness, self-similarity, and long-range
dependence. The Poisson process is unable to express any of these properties.
Another popular model is the fractional Gaussian noise \cite{lifshits2014},
which is a self-similar stochastic process. However, the process is suited for
modeling arrivals per unit of time but not for modeling arrival or inter-arrival
times, which are what is typically needed for simulation. Moreover, the noise
can take negative values and is a monofractal process, both of which are
unrealistic. The work in \cite{riedi1999} addresses the aforementioned concerns
by introducing a multifractal wavelet model that characterizes positive-valued
data with long-range-dependent correlations.

A performance simulator that we would like to highlight is Sniper
\cite{carlson2011}. Sniper is aimed at x86-based systems, and it has been
validated against Intel Core 2 and Nehalem architectures. The simulator works at
a higher level of abstraction than the one of cycle-accurate simulators, which
makes its simulation times more affordable. Regarding power simulation, a common
choice is the \sc{McPAT} framework \cite{li2009}. \sc{McPAT} is also capable of
estimating the areas occupied by processing elements, which is useful since this
information is essential for temperature simulation. Architecture-level
temperature simulation is arguably dominated by HotSpot \cite{skadron2004}. A
popular alternative is \sc{3D-ICE} \cite{sridhar2010}, which is particularly
focused on \sc{3D} structures.

The pipeline assembled from the aforementioned simulators is the one frequently
used in today's research. However, as we elaborate in \sref{introduction}, it is
extremely time consuming and, hence, is not suitable for learning purposes. In
our experience, this is mainly due to performance simulation and, to a lesser
extent, power simulation. Temperature simulation usually has a relatively small
cost compared to the other two. These concerns are to be discussed further and
illustrated in \sref{result}.

Our work makes the following major contribution. We present an efficient
toolchain for generating realistic power and temperature profiles of electronic
systems in order to facilitate the development of intelligent data-driven
techniques for the analysis and management of such systems. The toolchain is
open source and publicly available online at \cite{sources}.
