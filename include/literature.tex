The present work is related to computer simulation. For our purposes, it is
sufficient to distinguish four types of simulation: traffic, performance, power,
and temperature, which are shown in \fref{development}. In the scope of this
paper, \emph{traffic} refers to a stream of jobs that lade the system with work,
and \emph{performance} to the system's performance metrics such as the numbers
of executed instructions. We touch upon performance, power, and temperature in
this section and upon traffic in \sref{traffic}.

A performance simulator that we would like to highlight is Sniper
\cite{carlson2011}. Sniper is aimed at x86-based systems, and it has been
validated against Intel Core 2 and Nehalem architectures. The simulator works at
a higher level of abstraction than the one of cycle-accurate simulators, which
makes its simulation times more affordable. Regarding power simulation, a common
choice is the \sc{McPAT} framework \cite{li2009}. \sc{McPAT} is also capable of
estimating the areas occupied by processing elements, which is useful since this
information is essential for temperature simulation. Architecture-level
temperature simulation is arguably dominated by HotSpot \cite{skadron2004}. A
popular alternative is \sc{3D-ICE} \cite{sridhar2010}, which is more focused on
\sc{3D} structures.

The pipeline assembled from the aforementioned simulators is the one frequently
used in today's research. However, as we elaborate in \sref{introduction}, it is
extremely time consuming and, hence, is not suitable for learning purposes. In
our experience, this is mainly due to performance simulation and, to a lesser
extent, power simulation. Temperature simulation usually has a relatively small
cost compared to the other two. These concerns will be discussed further and
illustrated in \sref{result}.
