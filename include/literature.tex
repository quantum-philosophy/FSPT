The present work is related to simulation techniques. For our purposes, it is
sufficient to distinguish four types of simulation: traffic, performance, power,
and temperature, which are shown in \fref{development}. Here \emph{traffic}
refers to a stream of jobs that lade the system with work, and
\emph{performance} to the system's performance metrics such as the numbers of
executed instructions. We focus on performance, power, and temperature in this
section and on traffic in \sref{traffic}.

A performance simulator that we would like to highlight is Sniper
\cite{carlson2011}. Sniper is aimed at x86-based systems, and it has been
validated against Intel Core 2 and Nehalem architectures. The simulator works at
a higher level of abstraction than the one of cycle-accurate simulators, which
makes its simulation times more affordable. Regarding power simulation, a common
choice is the \sc{McPAT} framework \cite{li2009}. \sc{McPAT} is also capable of
estimating the areas occupied by processing elements, which is useful as this
information is essential for temperature simulation. Architecture-level
temperature simulation is arguably dominated by HotSpot \cite{skadron2004}. A
popular alternative is \sc{3D-ICE} \cite{sridhar2010}, which is more focused on
\sc{3D} structures.

The pipeline assembled from the aforementioned simulators is frequently used in
today's research. However, it is slow and stiff for learning purposes. This is
mainly due to performance simulation and, to a lesser extent, power simulation
as they are considerably time consuming. Temperature simulation has a relatively
small cost compared to the other two.

The need for a systematic assistance in obtaining data for learning-based
undertakings is prominent in the literature. For instance, the study in
\cite{lu2015} proposes a temperature-aware task-allocation strategy based on
reinforcement learning. The authors of that work had to invest time into
developing a custom simulation platform (still based on the above-mentioned
simulators) in order to experiment with their approach. From our perspective,
such auxiliary work should be streamlined.
