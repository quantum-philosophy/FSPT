\setlength{\tabcolsep}{4pt}
\begin{table}
  \begin{threeparttable}
    \caption{Streaming}
    \begin{tabular*}{\linewidth}{=L{50pt}-R{40pt}-R{40pt}-R{40pt}-R{40pt}}
      \toprule
      \multirow{2}{*}{\parbox{50pt}{Synthesized time (seconds)}} & \multicolumn{4}{c}{Synthesis time (seconds)} \\
      & $4 + 1$ & $8 + 2$ & $16 + 4$ & $32 + 8$ \\
      \cmidrule( r){1-1}
      \cmidrule(l ){2-5}
         10 &   0.24 &   0.40 &   0.67 &   1.13 \\
        100 &   2.11 &   3.66 &   6.18 &  10.19 \\
       1000 &  20.59 &  37.47 &  64.58 & 104.00 \\
      10000 & 214.98 & 394.84 & 598.89 & 984.59 \\
      \bottomrule
    \end{tabular*}
    \tlab{streaming}
    \begin{tablenotes}
      \item ``$M + N$'' stands for $M$ cores and $N$ L3 caches. Each four cores
        have a shared L3 cache; therefore, it holds that $N = M / 4$.
    \end{tablenotes}
  \end{threeparttable}
\end{table}
% vim: nowrap tw=0

Let us turn to Streamer, which corresponds to the data-synthesis stage (recall
\sref{streamer} and \fref{streamer}). Our objective in this subsection is to
study the scalability of the tool as measured by synthesis time, which is the
time that is needed for the tool to synthesize power and temperature profiles
under certain conditions or requirements. In these experiments, workload
patterns are assigned to job arrivals randomly, and a simple
first-in-first-served scheduling policy is assumed; both are the default
algorithms that come with the toolchain.

The preformed experiments are consolidated in \tref{streaming}. We report
synthesis time along two axes: synthesized time (rows) and platform size
(columns). The former is analogous to simulated time, and the latter represents
different platforms as follows. Each considered platform is composed of a number
of cores, and there is an L3 cache for every four cores; both cores and caches
are referred to as processing elements. Platform size is then defined as the
number of processing elements, and it is denoted by ``$M + N$'' in
\tref{streaming} where $M$ and $N$ are the numbers of cores and L3 caches,
respectively.\footnote{The specifications of the considered platforms are a part
of our supplementary materials available at \cite{sources}.}

Unlike the throughput of simulation (discussed in the previous subsection), the
throughput of synthesis in terms of synthesized time is very high, which is well
supported by \tref{streaming}. For instance, it takes Streamer around a minute
to produce power and temperature data that are worth around 17 minutes of
runtime of a multiprocessor system with 16 cores, which would be practically
infeasible to achieve with full-fledged simulations (refer to \tref{recording}).

Another observation made from \tref{streaming} is that synthesis time scales
linearly with respect to the length of the time span being synthesized
(synthesized time), and the same can be concluded regarding platform size. The
growth with respect to platform size is due to the increasing complexity of the
underlying thermal \sc{RC} circuit used for temperature simulation; thermal
circuits were elaborated on in \sref{streamer}.
