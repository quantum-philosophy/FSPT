The foremost step is construct a traffic model describing \emph{when} jobs
arrive. The model should satisfy a number of requirements in order to be
practically useful. First, it should be able to capture the idiosyncrasies
present in real traffic as it is a foundation for the subsequent generation of
power and temperature traces (discussed in \sref{composition}). Second, it
should be straightforward to configure given a data set of traffic data (a
sequence of arrival times of jobs) serving as a reference.

The required ability to absorb reference data acknowledges the importance of the
information that is at one's disposal due to monitoring and logging mechanisms
ubiquitously deployed nowadays. In \fref{methodology}, such mechanisms are
represented by a set of modules labeled ``Logger.'' The corresponding data are
are stored in a repository, which is accessible to the module labeled
``Streamer'' (to be discussed shortly).

Monitoring and logging mechanisms, which are a source of traffic patterns, are
outside of the scope of this work. For a good example of traffic data, the
interested reader is referred to the data set published by Google in 2011
\cite{google}; the data set will be discussed further in \sref{result}. Let us
now move on to the analysis and synthesis of a chosen traffic pattern.

The seminal work in \cite{leland1994} and subsequent studies have shown that
network traffic exhibits fractal properties such as burstiness, self-similarity,
and long-range dependence. It is well known that traffic models based on the
Poisson process \cite{lifshits2014} are unable to express any of the above
properties. A step in the right direction is the fractional Gaussian noise
\cite{lifshits2014}, which is a self-similar stochastic process. However, the
process is inconvenient to work with. More concretely, the noise is suited for
modeling arrivals per unit of time but not arrival or inter-arrival times, which
are what is typically needed for simulation. Moreover, the process can take
negative values, which is unrealistic. Finally, the fractional Gaussian noise is
a monofractal process; however, real traffic data often have multifractal
structures.

In order to address the aforementioned concerns, we employ the multifractal
wavelet model proposed in \cite{riedi1999}, which characterizes positive-valued
data with long-range-dependent correlations. In accordance with the model, we
take a reference time series of arrival times, analyze it by means of the
discrete wavelet transform based on Haar wavelets, and construct a certain
representation of the data, which can then be used for generating random time
series matching the fractal properties of the original one. As a result, the
model becomes tailored to the traffic pattern of the particular problem at hand.

To summarize, we have obtained a data set of reference arrival times and a
technique for analyzing them and generating similar arrival streams. The
technique is capable of capturing the properties that are commonly present in
real traffic.
