The foremost step is to construct a traffic model describing \emph{when} jobs
arrive. The model should satisfy a number of requirements in order to be
practically useful. First, it should be straightforward to configure given a
data set of traffic data (a sequence of arrival times of jobs) serving as a
reference. Second, it should be able to capture the idiosyncrasies present in
real traffic as it is a foundation for the subsequent generation of power and
temperature traces (discussed in \sref{composition}).

The required ability to absorb reference data acknowledges the utility of the
data that is readily at one's disposal due to the ubiquitously deployed
monitoring and logging mechanisms. These mechanisms are outside the scope of
this work. In \fref{methodology}, they are represented by a set of modules
labeled ``Logger.'' The collected traffic patterns are assumed to be stored in a
repository; see the top cloud in \fref{methodology}. For a good example of such
data, the interested reader is referred to the cluster-usage traces published by
Google in 2011 \cite{google}; the data set will be discussed further in
\sref{result}. Let us now move on to the analysis and synthesis of a particular
traffic pattern.

In order to be able to generate realistic traffic, we employ the multifractal
wavelet model proposed in \cite{riedi1999}, which is motivated in
\sref{literature}. In accordance with the model, we take a reference time series
of arrival times, analyze it by means of the discrete wavelet transform based on
Haar wavelets, and construct a certain representation of the data, which can
then be used for generating random time series matching the fractal properties
of the original one. As a result, the model becomes tailored to the traffic
pattern of the particular problem at hand.

To summarize, we have obtained a data set of reference arrival times and a
technique for analyzing them and generating similar arrival streams. The
technique is capable of capturing the properties that are commonly present in
real traffic.
