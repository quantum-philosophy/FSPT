As described in \sref{problem-formulation}, the system at hand serves a stream
of jobs, which can also be viewed as user requests. The foremost component of
our methodology is then a traffic model, describing how or, rather, when jobs
arrive. This model should satisfy a number of requirements in order to be
practically useful. First of all, the model should be able to capture well the
idiosyncrasies present in real traffic as it establishes a foundation for the
subsequent generation of power and temperature traces. Second, it should be
straightforward to configure the model given a dataset of arrival times serving
as a reference. This last requirement acknowledges the importance of the arrival
data that are at our disposal nowadays due to ubiquitously deployed monitoring
and logging systems.

The seminal work in \cite{leland1994} and subsequent studies have shown that
network traffic exhibits fractal properties such as burstiness, self-similarity,
and long-range dependence, which is very much unlike traditional telephone
traffic. It is well known that traffic models based on the Poisson process
\cite{lifshits2014} are unable to express any of the above properties and,
therefore, drastically departure from reality when it comes to modeling network
traffic. A step in the right direction is to consider the fractional Gaussian
noise \cite{lifshits2014}, which is a self-similar stochastic process. However,
the process is inconvenient to work with from the perspective of a computer
simulator. More concretely, the noise is suited for modeling arrivals per unit
of time but not arrival or inter-arrival times, which are what is typically
needed for simulation. Moreover, even with a proper rescaling and a positive
shift, the process can still take negative values, which is unrealistic.
Finally, the fractional Gaussian noise is a monofractal process; however, real
traffic data often have multifractal structures.

In order to address the aforementioned concerns and enable the generation of
arrival streams exhibiting fractal properties, our methodology employs the
multifractal wavelet model proposed in \cite{riedi1999}, characterizing
positive-valued data with long-range-dependent correlations. To elaborate, we
take a reference time series of arrival times, analyze it by means of the
discrete wavelet transform based on Haar wavelets, and construct a certain
representation of the data, which can then be used for generating random time
series matching the fractal properties of the original one. The model can be
tuned without any reference time series; however, the strength of the technique
is in the rigorous usage of reference data. By doing so, no manual parameter
tuning is needed, and the model gets tailored to the arrival pattern of each
particular problem.

To sum up, we now have a flexible technique for synthesizing arrival times,
which preserves the properties of real arrival data such as burstiness,
self-similarity, and long-range dependence.
