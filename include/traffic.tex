As stated in \sref{problem-formulation}, the system serves a stream of jobs,
which can also be viewed as user requests. The foremost component of our
methodology is then a traffic model, describing how or, rather, when jobs
arrive. Needless to mention, this model should capture well the idiosyncrasies
present in real traffic data as it establishes a foundation for our data
generation.

The most notable trait of network traffic is self-similarity
\cite{lifshits2014}, which manifests itself in many different ways. This
behavior is very much in contrast to the one exhibited by traditional telephone
traffic, for example. The seminal work in \cite{leland1994} brought attention to
this important aspect in the context of Ethernet \sc{LAN} traffic. Since then
self-similarity has been acknowledged and studied in other computer networks as
well.

It is well known that traffic models based on the Poisson process
\cite{lifshits2014}---in which the number of requests per unit of time follows
the Poisson distribution---are not capable of expressing self-similarity and,
therefore, drastically departure from reality when it comes to modeling network
traffic. One popular alternative is the fractional Gaussian noise
\cite{lifshits2014}, which is a self-similar process. However, the process is
inconvenient to work with from the perspective of a simulator. More concretely,
the noise is suited for modeling arrivals per unit of time but not arrival or
inter-arrival times, which are what is typically needed for simulation. This
implies the need for additional logic converting bin counts to time instances.
Moreover, the process can take negative values, which is nonrealistic. Finally,
the fractional Gaussian noise is a monofractal process; however, real traffic
data often have multifractal structures.

In order to generate realistic steams of job arrivals, we have decided to use a
multifractal wavelet model \cite{riedi1999}, which was originally proposed in
the context of network traffic modeling. In vain with other studies
\cite{nikitovic2004}. We use a dataset published by Google \cite{google}. The
dataset contains usage data of a computational cluster over a month period (May
2011).

After a lot of going back and forth, I found a paper (1999) that presented a
framework for the analysis and synthesis of network traffic based on a
multifractal wavelet model. The approach takes a (recorded, real) time series of
arrival times, analyzes it using the discrete wavelet transform, and fits a
model to that data that is able to generate synthetic traces matching the
properties (in particular, the multifractal structure) of the data. I thought it
was a good idea, and I implemented it.

Regarding the data needed for model fitting, you have probably seen papers
studying the so-called Google cluster dataset. It is a large dataset published
by Google in 2011 for research studies. It is a log of the traffic of one of
Google's clusters recorded over a month-time period (May 2011). I downloaded the
dataset and extracted what was relevant for us, that is, arrival times of the
jobs submitted to the cluster. Right now I am using this dataset for
multifractal wavelet modeling and subsequent generation of interarrival times.
The next step is scheduling and mapping.
