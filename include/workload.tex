In the previous subsection, we introduced our approach to synthesizing arrival
times, capturing the properties of real arrival data such as burstiness,
self-similarity, and long-range dependence. Now we need to associate a concrete
workload with each arrival time or, equivalently, with each job or user request.
These workloads should satisfy a number of requirements according to the goal of
this work. First of all, as emphasized throughout the paper, our goal is to
produce realistic power and temperature traces; therefore, the workloads should
represent well the applications/programs/services that the system is supposed to
provide to the end user. Second, a workload should be fast to evaluate, which,
in our context, refers to obtaining the power profile of that workload.

As motivated in \sref{introduction}, this is the most time-consuming part, and,
therefore, we would like to avoid it as much as possible.

In our experiments, we use the applications from two benchmark suites, namely,
from \sc{PARSEC} \cite{bienia2011} and \sc{SPEC CPU2006} \cite{cpu2006}.
