\documentclass[conference]{IEEEtran}
%
% General
%
\usepackage[english]{babel}

%
% Formatting
%
\usepackage{inconsolata}
\usepackage{hyperref}
\usepackage{lettrine}
\usepackage[factor=1700]{microtype}

\newcommand{\subscript}[1]{\text{\kern0.1em#1}}

%
% Figures
%
\usepackage{graphicx}

%
% Tables
%
\usepackage{array}
\usepackage{booktabs}
\usepackage{multirow}
\usepackage[flushleft]{threeparttable}

\newcolumntype{L}[1]{>{\raggedright\let\newline\\\arraybackslash\hspace{0pt}}m{#1}}
\newcolumntype{C}[1]{>{\centering\let\newline\\\arraybackslash\hspace{0pt}}m{#1}}
\newcolumntype{R}[1]{>{\raggedleft\let\newline\\\arraybackslash\hspace{0pt}}m{#1}}

\newcolumntype{=}{>{\global\let\currentrowstyle\relax}}
\newcolumntype{-}{>{\currentrowstyle}}

%
% Text shortcuts
%
\renewcommand{\sc}[1]{#1}

\newcommand{\ie}{i.e.}
\newcommand{\eg}{e.g.}

\renewcommand{\tt}[1]{\texttt{#1}}

%
% References
%
\newcommand{\eref}[1]{(\ref{equ:#1})}
\newcommand{\fref}[1]{Fig.~\ref{fig:#1}}
\newcommand{\sref}[1]{Sec.~\ref{sec:#1}}
\newcommand{\tref}[1]{Table~\ref{tab:#1}}

\newcommand{\elab}[1]{\label{equ:#1}}
\newcommand{\flab}[1]{\label{fig:#1}}
\newcommand{\slab}[1]{\label{sec:#1}}
\newcommand{\tlab}[1]{\label{tab:#1}}


\title{
  Massive Generation of High-Quality Power and Temperature Traces of
  Multiprocessor Systems
}

\author{\input{include/authors}}

\begin{document}
  \maketitle

  \begin{abstract}
    We present a methodology and a toolchain for the rapid generation of realistic
power and temperature traces of multiprocessor systems. The target audience of
this work is researchers who develop on-chip solutions that internally rely on
data-driven techniques in order to attain their objectives. Examples of use
cases include proactive power-, energy-, and temperature-aware management
strategies that leverage various machine-learning techniques. In this context,
the availability of a sufficiently large amount of data, which is essential for
learning and prediction and, consequently, for the exploration and exploitation
of research ideas, is rather elusive, to say the least. The presented
methodology and toolchain aim to fulfill this need, that is, to provide profuse
representative data to learn from at the development stage. The overarching goal
is to enable new and facilitate existing studies by making it tractable to
explore novel or revived, highly promising, but potentially data-demanding
techniques for modeling and prediction.

  \end{abstract}

  \begin{IEEEkeywords}
    Machine learning,
    multiprocessor systems,
    power,
    simulation,
    temperature,
    traffic,
    workload.
  \end{IEEEkeywords}

  \bstctlcite{IEEEexample:BSTcontrol}

  \section{Introduction} \slab{introduction}
  \lettrine[findent=0.4em, nindent=0em]{\textbf{P}}{power} consumption and heat
dissipation are of paramount importance. The two inseparable phenomena dictate
limitations on the usage of electronic devices and severely affect the costs
pertaining to the deployment and maintenance of electronic systems. Power is
essentially energy, and energy translates willingly to service times and
electricity bills. The situation is deteriorated by the fact that a higher level
of power consumption leads to higher temperatures, and higher temperatures
strike back by causing the device to consume even more power. Under these
circumstances, it is no surprise that power and temperature have been in the
limelight for a long time and have no plans on leaving this spot.

\cite{park2015}

Uncertainty.

Machine learning.

Neural networks.

Training data.

Real data.

Synthetic data.

Slow simulators.

Cycle-accurate simulation is an important design tool. It is particularly useful
for the design of individual processing units; however, cycle-accurate
simulation falls short when it comes to large multiprocessor systems. Such
systems are reasonably more complex, which leads to prohibitively large, often
infeasible, simulation times. On the other hand, in order to be properly
addressed, many questions asked in both academia and industry do not need cycle
accuracy. It is important to have an adequate level of abstraction in order to
stay focused on what matters the most to the problem at hand without being
constantly destructed by insignificant or unrelated issues. In such cases, cycle
accuracy can become a serious obstacle.

Sniper raises the level of abstraction \cite{carlson2011}.


  \section{Prior Work}
  Sniper \cite{carlson2011}.
  \sc{McPAT} \cite{li2009}.
  HotSpot \cite{skadron2004}.
  \sc{3D-ICE} \cite{sridhar2010}.

  \section{Our Contribution}

  \section{Modeling}
  \subsection{System}

  \subsection{Traffic}
  In order to generate realistic steams of job arrivals, we have decided to use
  a multifractal wavelet model \cite{riedi1999}, which was originally proposed
  in the context of network traffic modeling. In vain with other studies
  \cite{nikitovic2004}. We use a dataset published by Google \cite{google}. The
  dataset contains usage data of a computational cluster over a month period
  (May 2011).

  \subsection{Workload}
  \sc{PARSEC} \cite{bienia2011}.
  \sc{SPEC CPU2006} \cite{cpu2006}.

  \section{Simulation}
  \subsection{Performance}
  Sniper \cite{carlson2011}.
  \begin{table}
  \caption{Target architecture}
  \begin{tabular*}{\linewidth}{=L{70pt}l}
    \toprule
    Component    & Description \\
    \midrule
    Core         & 2660 MHz, 1.2 V \\
    L1-I/D cache & 32 KB, 4-way, LRU, private \\
    L2 cache     & 256 KB, 4-way, LRU, private \\
    L3 cache     & 8192 KB, 16-way, LRU, one per four cores \\
    \bottomrule
  \end{tabular*}
  \tlab{target}
\end{table}
% vim: nowrap tw=0


  \subsection{Power}
  \sc{McPAT} \cite{li2009}.

  \subsection{Temperature}
  HotSpot \cite{skadron2004}.
  \sc{3D-ICE} \cite{sridhar2010}.
  The solver is based on exponential integrators \cite{ukhov2012}.

  \section{Implementation}
  The tool has been implemented in the Rust programming language
\cite{rust}.\footnote{Rust is a systems programming developed by Mozilla
Research together with thousands of independent contributors from all around the
world. Rust features memory safety without garbage collection, concurrency
without data races, abstractions without overhead, and stability without
stagnation.} The tool is composed of a number of independent packages, which can
be used on their own. The source code of all the components is distributed under
the \sc{MIT} license and is available online \cite{sources}.

The key-value storage is Redis \cite{redis}. The database is SQLite
\cite{sqlite}.


  \section{Experimental Results}

  \section{Conclusion}
  In this paper, we have emphasized the need for developing tools for the design
of computer systems with data-driven applications in mind. We have argued that
the techniques which capitalize on learning from runtime data have special
requirements, and that the state-of-the-art simulators are unable to fulfill
them due to their prohibitively large simulation times.

Acknowledging the importance of power and temperature for the design of computer
systems, we have developed a methodology for fast generation of power and
temperature profiles of such systems, which preserve the idiosyncrasies of their
real-life counterparts. Following the methodology, we have implemented and
open sourced a toolchain, which has been assessed and shown to have a high
computational throughput.


  \section*{Acknowledgments}
  The authors would like to acknowledge everybody.


  \begingroup
    \bibliographystyle{IEEEtran}
    \bibliography{IEEEabrv,include/references}
  \endgroup
\end{document}
